%\begin{tabular}{|p{.5cm}|p{8.5cm}|}
\hline
&\textsc{Losses}\\
\hline
L1 & Room gets too cold (2 or more degrees below target)\\
\hline
L2 & Room gets too hot (2 or more degrees above target)\\
\hline
L3 & Damage to facilities, property, or the heating equipment itself\\
\hline
L4 & Waste of fuel\\
\hline
L5 & Physical harm to humans or pets\\
\hline
\end{tabular}
\vspace{1em}

\begin{tabular}{|p{.5cm}|p{8.5cm}|}
\hline
&\textsc{Losses}\\
\hline
L1 & Room gets too cold (2 or more degrees below target)\\
\hline
L2 & Room gets too hot (2 or more degrees above target)\\
\hline
L3 & Damage to facilities, property, or the heating equipment itself\\
\hline
L4 & Waste of fuel\\
\hline
L5 & Physical harm to humans or pets\\
\hline
\end{tabular}
\vspace{1em}

Both \emph{L1} and \emph{L2} reflect the notion of \emph{mission losses}, i.e. the system did not serve its core purpose. 

Similarly, we could include something like:
\marginnote{Compare this notion to ``service level agreements'' about system responsiveness.}

\begin{tabular}{ll}
L6&Unresponsiveness: The system is too slow to heat the room\\
L7&The system does not accept user input\\
\end{tabular}

From the perspective of a company manufacturing thermostats and furnaces, we might also choose to include other losses relevant to our business, such as:
\marginnote{Compare this to ``COGS'' concerns.}

\begin{tabular}{ll}
L8&System is excessively costly to manufacture\\
%L8&Failure to maintain safety certification from inspection boards\\
\end{tabular}

Alternatively, we might have as a loss, ``The heating system is financially unprofitable'', with ``excessively costly to manufacture'' as a hazard.
