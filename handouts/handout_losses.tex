 \documentclass[a4paper]{tufte-book}
\usepackage{booktabs}
\usepackage{tabularx}
\usepackage{longtable} 
\usepackage{lscape}
\usepackage{colortbl}

\geometry{
  left=.5in,
  right=.5in,
  top=.5in,
  bottom=.5in
}

\begin{document}

\begin{landscape}
\advance\vsize6cm
\csname @colroom\endcsname=\vsize
\textheight=\vsize
\csname @colht\endcsname=\vsize

\setlength{\parindent}{0em}
\setlength{\parskip}{.75em}


\begin{multicols}{2}
[ \section{Identifying Losses}]

\newthought{Definition}

Accident: An undesired or unplanned event that results in a \textbf{loss}, including loss of human life or human injury, property damage, environmental pollution, mission loss, etc. [\emph{Engineering A Safer World} p 181]

\textbf{Losses} are the outcomes we want to prevent.

Our task is to \textbf{write a list} of them, staying broadly general and covering all the areas of concern for our safety analysis to address.
\newthought{Thermostat Example}

%\begin{tabular}{|p{.5cm}|p{8.5cm}|}
\hline
&\textsc{Losses}\\
\hline
L1 & Room gets too cold (2 or more degrees below target)\\
\hline
L2 & Room gets too hot (2 or more degrees above target)\\
\hline
L3 & Damage to facilities, property, or the heating equipment itself\\
\hline
L4 & Waste of fuel\\
\hline
L5 & Physical harm to humans or pets\\
\hline
\end{tabular}
\vspace{1em}

\begin{tabular}{|p{.5cm}|p{8.5cm}|}
\hline
&\textsc{Losses}\\
\hline
L1 & Room gets too cold (2 or more degrees below target)\\
\hline
L2 & Room gets too hot (2 or more degrees above target)\\
\hline
L3 & Damage to facilities, property, or the heating equipment itself\\
\hline
L4 & Waste of fuel\\
\hline
L5 & Physical harm to humans or pets\\
\hline
\end{tabular}
\vspace{1em}

Both \emph{L1} and \emph{L2} reflect the notion of \emph{mission losses}, i.e. the system did not serve its core purpose. 

Similarly, we could include something like:
\marginnote{Compare this notion to ``service level agreements'' about system responsiveness.}

\begin{tabular}{ll}
L6&Unresponsiveness: The system is too slow to heat the room\\
L7&The system does not accept user input\\
\end{tabular}

From the perspective of a company manufacturing thermostats and furnaces, we might also choose to include other losses relevant to our business, such as:
\marginnote{Compare this to ``COGS'' concerns.}

\begin{tabular}{ll}
L8&System is excessively costly to manufacture\\
%L8&Failure to maintain safety certification from inspection boards\\
\end{tabular}

Alternatively, we might have as a loss, ``The heating system is financially unprofitable'', with ``excessively costly to manufacture'' as a hazard.

 \columnbreak

\newthought{Desired Qualities} 
\begin{itemize}
\setlength{\itemsep}{0pt}
\setlength{\parskip}{.25em}
\item Concise --- We want a relatively short list (<20); these are our priorities.
\item General --- We don't want to prematurely narrow our focus. 

Example: "Someone is injured" may be a more useful loss statement than "Someone is injured by hot equipment" because, overall, we want to prevent \emph{any} injury.
\item Good coverage --- For any accident we can think up, we want it to be described by at least one of the losses on this list.
\item Non-redundant --- Overlap between losses is ok, but if one loss is entirely a subset of another, perhaps consider consolidating them.
\item Relevant --- They should be problems we actually consider important to prevent for our system. 

Example: "Civil war breaks out" is not a loss relevant to our thermostat example (except jokingly, or if there's an allegory about climate change).
\end{itemize}  
 
\newthought{Strategic Approaches} 

Ask "what is \emph{unacceptable}?" vs. "what is \emph{risky but tolerable}?" to distinguish from hazards.

\emph{Toolkit}: List of generic losses.

\newthought{Relationship to other concepts}

For each of these \textbf{losses}, we will identify hazardous system conditions that, in combination with environmental conditions, can result in an accident in which we experience the loss.

The relationship between \textbf{losses} and \textbf{hazards} lets us \emph{prioritize} our safety efforts, focusing on preventing the system states that are relevant to producing these accidents--- we don't need to examine every combination of system states.

\end{multicols}
\end{landscape}
\end{document}